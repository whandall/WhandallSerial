\hypertarget{index_intro_sec}{}\section{Einführung}\label{index_intro_sec}
Die Verarbeitung seriellen Inputs bereitet vielen Arduino Anfängern Probleme.

Die niedrige Geschwindigkeit mit der die Zeichen eintreffen und das Single-\/\+Thread Umfeld des Arduinos erfordert eine für viele ungewohnte nicht-\/blockierende Programmierweise um nicht den gesamten Sketch zu blockieren.

Serielle Kommunikation wird so häufig benötigt und oftmals auch auf mehreren Schnittstellen gleichzeitig, daß sich die Erstellung einer Klasse in Form einer Arduino Library regelrecht aufdrängt.\hypertarget{index_hist_sec}{}\section{History}\label{index_hist_sec}

\begin{DoxyItemize}
\item 2017 Erweiterung auf Nextion Format und damit binäre Daten
\item 2017 erste Version C\+R/\+LF -\/ A\+S\+C\+II
\end{DoxyItemize}\hypertarget{index_install_sec}{}\section{Installation}\label{index_install_sec}
Standard Arduino Library Installation 